\section{Adaptive Framework (MLOps + Informer) for Traffic Prediction in Edge Computing}

This section explores an innovative adaptive framework for predicting network traffic in edge computing environments. It integrates Edge MLOps \cite{9610376, 10188244} with tools like MLflow, Seldon, and MinIO S3 buckets, and employs Dynamic Time Warping (DTW) for effective clustering of traffic data \cite{Downey2023, Tuna2023, Soysal2023}.

\begin{itemize}
    \item \textbf{Framework Overview:} The framework uses informers and DTW-based clustering for accurate traffic prediction in edge computing scenarios. It leverages MLflow for model tracking, Seldon for model serving, and MinIO as an S3-compatible storage solution \cite{Downey2023}.
    
    \item \textbf{DTW-Based Clustering:} Dynamic Time Warping (DTW) is utilized for clustering network traffic data, enabling the framework to capture and adapt to varying traffic patterns efficiently \cite{Tuna2023}.
    
    \item \textbf{MLOps Integration:} Integrating MLOps principles streamlines the machine learning lifecycle, encompassing model development, testing, and deployment. This ensures efficient model management and rapid deployment in dynamic edge environments \cite{Soysal2023}.

    \item \textbf{Model Selection Using Real-Time Data:} MLflow plays a critical role in model deployment. By attaching datasets as artifacts, it allows for comparison of real-time data with existing models. This enables the selection of the most appropriate model based on current network conditions \cite{Downey2023}.

    \item \textbf{CI/CD Integration for Model Updates:} The framework incorporates a Continuous Integration/Continuous Deployment (CI/CD) pipeline, allowing for seamless model updates and shifts without disrupting ongoing inference processes. This ensures that model enhancements and adaptations are deployed efficiently, maintaining uninterrupted service in dynamic edge computing environments. 

    \item \textbf{Real-Time Traffic Prediction:} Advanced machine learning techniques, including informers, are employed for real-time traffic prediction, improving the responsiveness and accuracy of traffic management in edge computing networks \cite{Downey2023}.
    
    \item \textbf{Adaptability and Scalability:} The framework is designed to adapt to changing network conditions and scale efficiently, ensuring consistent performance and reliability in diverse and uncertain edge environments \cite{Tuna2023, Soysal2023}.
\end{itemize}

\subsection{Related Work}
Downey et al. \cite{Downey2023} initially proposed an adaptive framework using time-series analysis for network traffic prediction. Tuna and Soysal \cite{Tuna2023, Soysal2023} extended this work, focusing on the integration of MLOps and the application of DTW for clustering in traffic prediction within edge computing scenarios.

