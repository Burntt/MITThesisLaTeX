\section{Cellular Networks}

This section discusses the application of deep learning for spatiotemporal modeling in cellular networks, following the approach by Wang et al. \cite{Wang2017Spatiotemporal}. The study utilizes extensive cellular network data, highlighting the effectiveness of deep learning techniques in network prediction tasks. Key highlights include:

\begin{itemize}
    \item \textbf{Preliminary Data Analysis:} An initial analysis on a large dataset from a major wireless carrier, showcasing significant temporal and spatial dependencies in network usage \cite{Wang2017Spatiotemporal, Zhang2017Spatiotemporal}.
    
    \item \textbf{Hybrid Deep Learning Model:} Introduction of a hybrid model combining autoencoder-based spatial modeling with LSTM units for temporal modeling, enhancing prediction accuracy significantly over traditional models like ARIMA \cite{Wang2017Spatiotemporal, Hochreiter1997, Box2015ARIMA}.
    
    \item \textbf{Performance Evaluation:} The proposed model was thoroughly evaluated, showing substantial improvements in prediction accuracy compared to traditional methods such as ARIMA and SVR \cite{Wang2017Spatiotemporal, Drucker1997SVR}.
    
    \item \textbf{Applications and Implications:} This method has significant implications for optimizing resource allocation and managing network loads in real-time, showcasing the potential of deep learning in telecommunications \cite{Wang2017Spatiotemporal, Goodfellow2016DeepLearning}.
\end{itemize}
