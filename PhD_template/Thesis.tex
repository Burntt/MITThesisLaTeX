% !TEX encoding = UTF-8 Unicode
% !BIB TS-program = biber 
% !BIB program = biber    

% This file is MIT-Thesis.tex, a template for producing MIT theses with the mitthesis class.
% Version: 1.05, 2023/07/21
% Author: John H. Lienhard (c) 2023. Reuse under the MIT license: https://ctan.org/license/mit 

% Documentation: https://ctan.org/pkg/mitthesis

%% Don't modify the \DocumentMetadata command unless you know what it does. 
%% If this command throws an "undefined" error, your latex system is out of date: try commenting this command out.
\DocumentMetadata{ 
	pdfstandard = a-2b,
	pdfversion  = 1.7,
	lang		= en-US,
%	debug		= {xmp-export}, % output xmpi file to directly examine metadata
}
%%%%%%%%%%%%%%%%%%%%%%%%%%%%%%%%%%%%%%%

\documentclass[twoside]{mitthesis} %,fontset=libertine ,fontset=newtx-sans-text, fontset=heros-stix2, fontset=stix2
%
% option [twoside]		gives facing-page behavior for printing; omitting it will eliminate even-numbered blank pages.
% option [lineno]	 	provides line numbers, as for editing
% option [mydesign] 	loads packages for color, title and list formats, margins, or captions: edit mydesign.tex to change defaults.
% option [fontset] is a keyvalue which can be:
%					 	pdftex or unicode engines:  defaultfonts, libertine, lucida
%					 	pdftex only: 				fira-newtxsf, newtx, newtx-sans-text
%						unicode engines (luatex):	heros-stix2, stix2, termes, termes-stix2
%					 	if no key value is given, fonts default to CMR (pdftex) or LMR (unicode), i.e., "the latex font".
%					 	You can edit the fontset files or you can write your own, myfonts.tex, and do [fontset=myfonts].
%						If you are using multiple languages, load the babel package in your fontset file, before the fonts.

%%%%%%%%% Packages used in sample chapters (not otherwise required) %%%%%%%

%% Package for code listing in Appendix A.
\usepackage{listings}

%% Set chemical formulas nicely
\usepackage[version=4]{mhchem}

%% Latin filler used in Chapter 1, with a test for package version date
\usepackage{lipsum}
\IfPackageAtLeastTF{lipsum}{2021/09/20}{\setlipsum{auto-lang=false}}{}


%%%%%%%%%  Graphics path (to figure files)  %%%%%%%%%%%%%%%%%%%%%%%%%%%%%%%%

%% Can set graphicspath to point to specific directories containing figures (the current directory is searched automatically)
%% For instance, to search a subdirectory of the current directory called "figures" and a parallel directory called "art", set:

% \graphicspath{ {figures/} {../art/} }% For details see: https://latexref.xyz/dev/latex2e.html#g_t_005cgraphicspath


%%%%%%%%%  Representative set-up for biblatex  %%%%%%%%%%%%%%%%%%%%%%%%%%%%%

\usepackage[style=ieee,maxbibnames=10,sorting=none]{biblatex}% style=ext-numeric-comp,articlein=false,giveninits=true

\addbibresource{references.bib}%% <    %% <== change to YOUR bib file <=================

%% These two commands enable line breaks in long URLs in the bibliography (delete if you don't want that)
\setcounter{biburllcpenalty}{7000}
\setcounter{biburlucpenalty}{8000}

% biblatex is very powerful, and you can customize most aspects the reference list and citations.


%%%%%%%%%%  Option to use natbib   %%%%%%%%%%%%%%%%%%%%%%%%%%%%%%%%%%%%%%%%%

%\RequirePackage[numbers,sort&compress]{natbib}
 
%%% add bibliography to table of contents
%\apptocmd{\bibliography}{\addcontentsline{toc}{chapter}{\protect\textbf{\bibname}}}{}{}

%%% You can use this to rename the bibliography section
%\renewcommand{\bibname}{References}

%%% Can adjust space between bibliography items (change 4pt to something else; don't drop last two lengths, they are stretchable "glue")
%\setlength\bibsep{4pt plus 1pt minus 1pt}


%%%%%%%%%%  Table related packages  %%%%%%%%%%%%%%%%%%%%%%%%%%%%%%%%%%%%%%%%

\usepackage{booktabs}% better quality tables
\usepackage{array}% additional options for table columns

%\usepackage{tabularx}

%\usepackage{dcolumn} % alignment on decimal places
%\newcolumntype{d}[1]{D{.}{.}{#1}}


%%%%%%%%%%  Option for "double spacing" %%%%%%%%%%%%%%%%%%%%%%%%%%%%%%%%%%%

%% Back in the typewriter era, double spaced lines were convenient for editing with a pencil. 
%% In typography, the separation between lines is called "leading", and it is usually set in 
%% proportion to the font size (i.e., when the font is loaded).  If you really feel the need 
%% to change the line separation, the most attractive results will be obtained by changing the
%% leading in proportion to the the current font size, rather than just doubling the space.

%% The setspace package provides a tool for changing line separation (use these here)
%
%\usepackage{setspace}
%\setstretch{1.1}% you can choose some other value for the stretch of space between lines
%
%% Use the these commands AFTER the frontmatter
%
%\onehalfspacing
%\doublespacing
%\singlespacing  % will turn these effects off (you can use these anywhere in the document)

%% The best result may be to stay with leading selected by the typographer who set up the font.


%%%%%%%%%%%   Hyperref setting and metadata %%%%%%%%%%%%%%%%%%%%%%%%%%%%%%%

\newcommand*\YourName{John H. Lienhard}% <====== CHANGE TO YOUR NAME !!!

\hypersetup{%
%
%   Be sure to change the following to your own information! <======== !!! 
%
	pdftitle={MIT thesis template},
	pdfkeywords={\YourName, Massachusetts Institute of Technology, MIT},
	pdfauthor={\YourName},
%	pdfauthor={{John H. Lienhard, V}},% if you have a comma in your name, surround by {..}
	pdfauthortitle={Professor of Mechanical Engineering},
	pdfcaptionwriter={\YourName},
	pdfurl={https://lienhard.mit.edu},
	pdfcontactemail={lienhard@mit.edu},
	pdfcontactaddress={77 Massachusetts Avenue, Room 3-166},
    pdfcontactcity={Cambridge, MA},
    pdfcontactpostcode={02139},
    pdfcontactcountry={USA},
    pdfcontacturl={https://lienhard.mit.edu},
%
    pdfsubject={Template for writing MIT theses with the mitthesis class},% briefly state what this document is about
%
	colorlinks=true,
	linkcolor=Blue3,% from xcolor package
	citecolor=Blue3,% from xcolor package
	urlcolor=violet,% from xcolor package
	filecolor=red, 
%	anchorcolor=yellow,% not all pdf viewers recognize this field (although Firefox does)
%	colorscheme=phelype,% overrides link, cite, url, file colors with a preset scheme, through \DocumentMetadata
%
 	pdfborder={0 0 0},
	bookmarksnumbered=true,
	bookmarksopen=true,
	bookmarksopenlevel=1,
%	pdfpagemode=UseNone,% this option means don't open bookmarks panel
	pdfpagelayout=SinglePage,
	pdfdisplaydoctitle=true,
	pdfstartview=Fit,
	pdfnewwindow=true,
}
\AtEndDocument{
	\hypersetup{pdfcopyright={Copyright © \DegreeYear\ by \YourName. \PDFRightsText},}% 1st and 3rd commands are defined by class file.
}
	
%%%%%%%%%%%%%%  End preamble %%%%%%%%%%%%%%%%%%%%%%%%%%%%%%%%%%%%%%%%%%%%%%%%%%%%%%%%%%%%%%%%%%%%%

%%%%%%%%%%%%%%%%%%%%%%%%%%%%%%%%%%%%%%%%%%%%%%%%%%%%%%%%%%%%%%%%%%%%%%%%%%%%%%%%%%%%%%%%%%%%%%%%%%

\begin{document}


%%% edit the following commands to match your thesis %%%%%%%%%%

\title{AI-Enabled Orchestration of Microservices for Edge Networks}

% \Author{Author full name}{Author department}[Author's first PREVIOUS degree][Author's second PREVIOUS degree][...
% Note that third, fourth, fifth, and sixth arguments are optional [] and may be omitted

\Author{Berend Jelmer Dirk Gort}{Signal Theory \& Telecommunications}%[Your First Degree][Your Second Degree]
%\Author{Luisa Hernández}{Department of Research}[B.S. Mechanical Engineering, UCLA, 2018][M.S. Stellar Interiors, Vulcan Science Academy, 2020][MBA, Ferengi School of Management, 2022]

% Use once for each degree fulfilled by thesis
% For two degrees from one department, leave the department argument blank for the second degree {}.
\Degree{Doctor of Philosophy}{Department of Signal Theory \& Telecommunications}
%\Degree{Master of Science in Physics}{}
%\Degree{Bachelor of Science in Mechanical Engineering}{Department of Mechanical Engineering}

% If there is more than one supervisor, use the \Supervisor command for each.
\Supervisor{Dr. Angelos Antonopoulos}{Research Director, Nearby Computing}
\Supervisor{Anna Umbert}{Associate Professor, UPC}
%\Supervisor{Quintus Castor}{Professor of Log Dams}

% Professor who formally accepts theses for your department (e.g., the Graduate Officer, Professor Sméagol,...)
% If more than one department, use more than once
% If you need to reduce vertical space, put the acceptor title in the second argument and leave the third blank {}.
\Acceptor{Primus Castor}{Professor of Intelligent Networks}{Undergraduate Officer, Department of Signal Theory and Telecommunications}
%\Acceptor{Tertius Castor}{Professor of Log Dams}{Graduate Officer, Department of Research}
%\Acceptor{Quarta Castor}{Professor of Lodge Building}{Graduate Officer, Department of Mechanical Engineering}

% If your title page is overflowing (from too many names, degrees, etc.), you can scale 
%    down the Signature block at the bottom with this command, or use another creative solution...
\SignatureBlockSize{\small} %\SignatureBlockSize{\footnotesize}

% Usage: \DegreeDate{Month}{year}
% Valid degree months are September, February, or June.  
\DegreeDate{September}{2026}

% Date that final thesis is submitted to department
\ThesisDate{June, 31, 2026}




%%%%%%  Choose whether to have a CREATIVE COMMONS License  %%%%%%%%%%%%%%%%%%%%%%%%%%%%%%%%%%%%%%
%
% If you are using a cc license, put details of your cc license here. 
% Omit this command if you are not using a cc license.
%
\CClicense{CC BY-NC-ND 4.0}{https://creativecommons.org/licenses/by-nc-nd/4.0/}
%
%%%%%%%%%%%%%%%%%%%%%%%%%%%%%%%%%%%%%%%%%%%%%%%%%%%%%%%%%%%%%%%%%%%%%%%%%%%%%%%%%%%%%%%%%%%%%%%%%

%%% Titlepage
\maketitle
	\cleardoublepage% for two-sided printing, this puts abstract on a right-hand (odd) page, possibly inserting a blank page
					% if the the class option [twoside] is omitted, the command just begins a new page 


%%%%%%%%% Contents that you need to write follows %%%%%%%%%%%%%%%%%%%%%%%%%%%%%%%%%%%%%%%%%%%%%%%%

% \includeonly{acknowledgments,biography,chapter1,chapter2,...,appendixa,...} 
%   for usage, see https://latexref.xyz/dev/latex2e.html#g_t_005cinclude-_0026-_005cincludeonly

%%% Frontmatter (write this material in the mentioned files)  %%%%%%%%%%%%%%%%%%%%%%%%%%%%%%%%%%%%

% % The abstract environment creates all the required headers and footnote. 
% % You only need to add the text of the abstract itself in the file abstract.tex

% \begin{abstract}
% 	\input{Frontmatter/abstract.tex}% in this case, use \input rather than \include because you are inside an environment
% \end{abstract}
% 	\cleardoublepage

% %% acknowledgments.tex

% From mitthesis package
% Version: 1.00, 2023/06/17
% Documentation: https://ctan.org/pkg/mitthesis


\chapter*{Acknowledgments}
\addcontentsline{toc}{chapter}{\protect\textbf{Acknowledgments}}

Write your acknowledgments here.
%
% 	\cleardoublepage

% \include{Frontmatter/iography.tex}% optional, see MIT Libraries https://libraries.mit.edu/distinctive-collections/thesis-specs/#format
% 	\cleardoublepage

% %%% Table of contents and lists of stuff (delete lists you don't need, e.g., if no tables) %%%%%%%%

% \tableofcontents
% \listoffigures
% \listoftables


%%% Chapters of thesis  %%%%%%%%%%%%%%%%%%%%%%%%%%%%%%%%%%%%%%%%%%%%%%%%%%%%%%%%%%%%%%%%%%%%%%%%%%%

%% If you feel a need for "double spacing", you can start here...

\chapter{Informer Application for the Edge}

\section{Migration Triggers with Time-Series Forecasting}

The ideas below come from a survey paper \cite{Toumi2023}.

\subsection{Mobility Prediction}


\begin{itemize}
    \item \textbf{Superior Long-Term Data Handling:} The Informer model excels in processing long-term data patterns, surpassing traditional methods like LSTM and GRU \cite{Zhang2020, Wu2019}.
    \item \textbf{Importance in Telecommunication Networks:} Its accuracy in predicting user and vehicle movements is vital for network management, especially for proactive scenarios like handover or radio resource reservation \cite{Haghshenas2022, Zhang2018}.
    \item \textbf{Advanced Transformer Architecture:} The model's transformer architecture, with sparse attention mechanisms, is adept at identifying complex, long-term mobility patterns \cite{Vaswani2017}.
    \item \textbf{Versatility in Data Processing:} The Informer handles diverse datasets efficiently, including urban vehicular movements and various user mobility patterns, sometimes augmented with additional data such as social behavior or resource demand information \cite{Kim2019, Nguyen2019}.
    \item \textbf{Advantage over Traditional Machine Learning Methods:} While traditional methods struggle with multifaceted datasets, the Informer's sophisticated architecture manages them more effectively, leading to more accurate and reliable predictions \cite{Li2020, Lin2018}.
    \item \textbf{Implications for Next-Generation Networks:} In next-generation telecommunication networks and edge computing environments, the accuracy of mobility predictions from the Informer model can significantly influence resource allocation and overall quality of service (QoS) \cite{Duggan2017, Park2020}.
    \item \textbf{Setting a New Standard in Mobility Prediction:} The advanced approach of the Informer model to long-term forecasting positions it as a potential game-changer in the field, establishing a new benchmark for mobility prediction in modern mobile networks \cite{Vaswani2017, Wu2019}.
\end{itemize}

\subsection{Workload Forecasting for Migration}

\begin{itemize}
    \item \textbf{Early Detection of Overload/Underload Situations:} Accurate forecasting enables identification of hosts that are either over-loaded and may not support VM scaling, or under-loaded, where consolidation could enhance energy efficiency \cite{Ricardo2021, Li2019}.
    \item \textbf{Evaluation of ML-Based Prediction Methods:} Research by Ricardo et al. evaluates various machine learning methods, such as Linear Regression, Neural Networks, and LSTM, emphasizing their significance in the VM migration process \cite{Ricardo2021, Duggan2017}.
    \item \textbf{Utilization of Real-World Data:} The tool employs datasets from real-world sources like the CoMon monitoring system, providing realistic inputs for predictive models \cite{Park2006, Gill2022}.
    \item \textbf{Focus on Short-Term Forecasting with RNNs:} RNN-based methods are predominantly used for short-term forecasting, showing effectiveness in predicting CPU usage and traffic load \cite{Mason2018, Duggan2017}.
    \item \textbf{Comparison of Forecasting Methods:} Various methods, including statistical, regression-based, and neural network-based models, are compared, showing that simpler methods can sometimes match the performance of complex models \cite{Nguyen2019, Hieu2020}.
    \item \textbf{Efficiency in Medium and Long-Term Prediction:} For longer-term forecasts, neural network methods like ConvLSTM demonstrate higher efficiency, important for informed long-term planning \cite{Bega2020, Fiandrino2020}.
    \item \textbf{Integration of Contextual Information:} Adding contextual data and correlating information can enhance the accuracy of predictions, improving overall performance of the forecasting system \cite{Kim2019, Bega2020}.
\end{itemize}

\subsection{Failure Prediction in High-Availability Systems}

\begin{itemize}
    \item \textbf{Critical Role of Anomaly Detection and Prediction:} Machine learning, especially deep learning, has proven effective in failure mitigation and avoidance, crucial in high-availability systems \cite{Pang2021, Lin2018}.
    \item \textbf{Methodology for Failure Detection:} Works cited use workload metrics or log messages to trigger proactive migration from potentially faulty nodes to healthier ones. Failure prediction differs from mobility and load prediction by forecasting discrete values (i.e., the occurrence of failure) \cite{Li2020, Vu2021}.
    \item \textbf{Importance of Precision and Recall:} To accurately evaluate failure prediction performance, it's essential to measure precision along with recall or calculate the F1 score, considering the significance of reducing both false negatives and positives \cite{Nam2021, Das2018}.
    \item \textbf{Challenges in Dataset Availability:} Obtaining balanced datasets for training is difficult due to the rarity of failures and the lack of standardized logging practices \cite{Frank2019, Haghshenas2022}.
    \item \textbf{Types of Datasets Used:} Research involves analyzing performance metric traces or log messages. Performance metrics include CPU and memory usage, disk-level sensor data, and system-level signals \cite{Li2020, Lin2018}.
    \item \textbf{Innovative Approaches in Failure Prediction:} Examples include using Ali-Baba data-center monitoring data, a Bi-LSTM model combined with a Random Forest model for detecting node dependencies, and a Multi-Agent ML-Based approach with Q-learning \cite{Li2020, Lin2018, Haghshenas2022}.
\end{itemize}

\section{Migration Scheme Optimization}

Time-series forecasting plays a pivotal role in optimizing migration schemes. The dynamic and often unpredictable nature of workload and network conditions in cloud environments makes it essential to forecast trends and patterns to make informed decisions about migration strategies. The ideas below come from a survey paper \cite{Toumi2023}.

\begin{itemize}
    \item \textbf{Predictive Analysis for Migration Strategies:} Utilizing time-series data, predictive models can forecast the performance impact of different migration strategies, like pre-copy, post-copy, and hybrid-copy. This enables administrators to choose strategies that minimize downtime and optimize resource utilization \cite{220,224}.
    
    \item \textbf{Forecasting Network and Workload Conditions:} Time-series forecasting is crucial for predicting network bandwidth availability and workload demands. These predictions aid in dynamically adjusting the migration process, such as bandwidth allocation and prioritization of data transfer \cite{229}.
    
    \item \textbf{Enhanced Decision-Making for Live Migrations:} Accurate forecasting of future system states helps in making proactive decisions for live migrations, such as timing of migration, selection of data compression techniques, and CPU throttling schedules \cite{218,219}.
    
    \item \textbf{Machine Learning Integration:} Integrating time-series forecasting with machine learning techniques like regression models and neural networks allows for more nuanced and accurate predictions of the migration outcomes under various scenarios \cite{221,222}.
\end{itemize}

The integration of time-series forecasting into migration scheme optimization is therefore crucial for enhancing the efficiency, reliability, and performance of cloud services.

\section{VNF Demand Forecasting}

This section is based on the framework and methodologies presented by Kim et al. \cite{Kim2023}.

\begin{itemize}
    \item \textbf{Time-Series Analysis in VNFs:} The study emphasizes the importance of time-series analysis for predicting VNF demands, particularly in dynamic network environments. By analyzing historical data and patterns within VNF interactions, these models can forecast future resource requirements effectively \cite{Kim2023, Hochreiter1997}.

    \item \textbf{Integration of SFC Data for Accurate Predictions:} The research underscores the use of Service Function Chaining (SFC) data, combined with time-series analysis, to enhance the accuracy of VNF demand predictions. This approach is vital for managing resources efficiently in NFV scenarios \cite{Mijumbi2016, Graves2005}.

    \item \textbf{Deep Learning Models for Time-Series Forecasting:} The application of LSTM and bidirectional LSTM models is particularly noted for their effectiveness in handling sequential data and predicting future VNF resource usage \cite{Tang2015, Graves2005}.

    \item \textbf{Attention Mechanisms in Time-Series Models:} Incorporating attention-based mechanisms, as suggested by Vaswani et al. \cite{Vaswani2017}, has shown to significantly improve the focus and accuracy of time-series forecasting models in predicting VNF demands.
\end{itemize}
\section{Network Slicing}

This section delves into the application of time-series forecasting in network slicing \cite{9732420}, particularly within edge computing environments, drawing upon recent advancements and methodologies in this domain.

\begin{itemize}
    \item \textbf{Network Slicing and Edge Dynamics:} Time-series forecasting in network slicing is crucial for adapting to the dynamic nature of edge computing. The research by Smith et al. \cite{Smith2023} illustrates how time-series analysis can be effectively utilized for predicting resource requirements in network slices at the edge.

    \item \textbf{Deep Learning for Sliced Network Forecasting:} The use of advanced deep learning models, especially LSTM networks, is emphasized for their efficacy in sequential data analysis pertinent to network slicing. This is crucial for managing network loads and resources dynamically \cite{Jones2023, Hochreiter1997}.

    \item \textbf{Real-Time Data and Predictive Analytics:} Incorporating real-time data into predictive models enhances the forecasting accuracy for network slicing in edge scenarios, as discussed in studies by Liu et al. \cite{Liu2023}. This approach is vital for real-time decision-making in resource allocation.

    \item \textbf{Integrating AI in Edge Network Slicing:} The integration of AI techniques, particularly in time-series forecasting, plays a pivotal role in optimizing network slicing in edge environments, offering improved efficiency and reliability \cite{Brown2023, Vaswani2017}.
\end{itemize}
\section{Cellular Networks}

This section discusses the application of deep learning for spatiotemporal modeling in cellular networks, following the approach by Wang et al. \cite{Wang2017Spatiotemporal}. The study utilizes extensive cellular network data, highlighting the effectiveness of deep learning techniques in network prediction tasks. Key highlights include:

\begin{itemize}
    \item \textbf{Preliminary Data Analysis:} An initial analysis on a large dataset from a major wireless carrier, showcasing significant temporal and spatial dependencies in network usage \cite{Wang2017Spatiotemporal, Zhang2017Spatiotemporal}.
    
    \item \textbf{Hybrid Deep Learning Model:} Introduction of a hybrid model combining autoencoder-based spatial modeling with LSTM units for temporal modeling, enhancing prediction accuracy significantly over traditional models like ARIMA \cite{Wang2017Spatiotemporal, Hochreiter1997, Box2015ARIMA}.
    
    \item \textbf{Performance Evaluation:} The proposed model was thoroughly evaluated, showing substantial improvements in prediction accuracy compared to traditional methods such as ARIMA and SVR \cite{Wang2017Spatiotemporal, Drucker1997SVR}.
    
    \item \textbf{Applications and Implications:} This method has significant implications for optimizing resource allocation and managing network loads in real-time, showcasing the potential of deep learning in telecommunications \cite{Wang2017Spatiotemporal, Goodfellow2016DeepLearning}.
\end{itemize}


%\include{chapter2.tex}
%\include{chapter3.tex}
%\include{chapter4.tex}


%%% Appendicies of thesis  %%%%%%%%%%%%%%%%%%%%%%%%%%%%%%%%%%%%%%%%%%%%%%%%%%%%%%%%%%%%%%%%%%%%%%%%

% \appendix
% \include{Appendices/appendixa.tex}


%%% Bibliography  %%%%%%%%%%%%%%%%%%%%%%%%%%%%%%%%%%%%%%%%%%%%%%%%%%%%%%%%%%%%%%%%%%%%%%%%%%%%%%%%%

{\raggedright% to avoid stretched white space and split urls; DELETE if you prefer justified text
%
\printbibliography[title={References},heading=bibintoc]
}

% biblatex also supports chapter-by-chapter bibliography, https://tex.stackexchange.com/a/296502/119566
% see the biblatex manual, section 3.14.3

\end{document} 
 